\chapter{データ取扱いの心得}

\begin{chapintro}
    まず本章で、データの取り扱いについて学びましょう。例えば、健康診断では、身長と体重を測り、これらをもとに健康の指標としてBMI 指数を計算し、肥満度を考察します。ここで、身長、体重、BMI 指数がデータであり、これらのデータを分析することで肥満度が推論できます。この章で、データ を取得してから推論するまでの全体的なイメージを押さえてください。ビッグデータ? 時代にしっかり対応できるよう、データ取り扱いの基本を学んでいきましょう。
\end{chapintro}

\section{何のための確率・統計か}
学歴社会の是非はともかく、多くの日本の高校生が大学受験に際して偏差値(117 ページ)に振り回されています。最後の模擬試験でとった偏差値が58の人は、偏差値 65 が必要の希望の国立大学を受けるか、ランクを下げるか迷うこともあるでしょう。偏差値の高い大学を卒業すれば、幸せになれるとは限りませんが、入れる大学に入るという考え方も本末転倒かもちしれません。

第一志望の大学に何浪してでも入りたいという信念があれば迷わないのでしょうが、 実際にはいろいろな間題がのしかかってきます。 妥協したからとい\nobreak って、一概に責められるべきではないでしょう。

ここで、妥協の根拠は、覚悟(納得)するうえで重要です。例えば、家が貧しく給料のよい会社に早く就職したいという事情があれば、 合格可能な範囲で就職有利な大学(得てして偏差値の高い大学)を受験する決断ができます。もちろん必ず入試に受かるとは限りませんので、家計状況や社会経済状況などを勘案し、どの大学を受けるか、最悪一浪は我慢するか、などと検討をた重ねるこ
とになります。

このように、人生ではとかく情報に基づいて判断することを求められます。さらに、AIが台頭した昨今、人間が頭に蓄えている\uemph{知識自体の価値は下がり}、専門家ですら AIに適わないと言われ始めています。このような時代に適応して、\uemph{人間でないとできないこと、すなわち、課題を設定し、それに対して情報を入手し覚悟をもって判断することこそ、できるようになりたいものです}(\sfcref{fig:chart})。

\begin{figure}[h]
    \centering
    \includegraphics[width=.95\textwidth,height=.3\textwidth]{example-image-a}
    \caption{情報を用いた問題解決への道(模式図)\label{fig:chart}}
\end{figure}

\begin{simQ}
    過酷な環境で使う機械を、1種類の材料だけで作るとします。調査の結果、最終候補材は材料AとBになりました。これらについて、主要な四つの強さの指標を測定し、{\sfcref{tab:mat}}の結果を得ました。そして、互いに比較できるよう偏差値に換算しました。
    
    \begin{center}\footnotesize
        \captionof{table}{強さ試験の結果\label{tab:mat}}
        \rowcolors{1}{white}{white}
        \begin{tabular}{crr}
            \specialrule{1.2pt}{0pt}{0pt}
            & 材料A & 材料B \\ 
            \specialrule{.6pt}{0pt}{0pt}
            韧性 & 66 & 62 \\
            硬度 & 71 & 75 \\
            耐食性 & 69 & 64 \\
            破断強度 & 68 & 73 \\ 
            \specialrule{.6pt}{0pt}{0pt}
            平均 & 68.5 & 68.5 \\ 
            \specialrule{1.2pt}{0pt}{0pt}
        \end{tabular}
    \end{center}
    
    \begin{enumerate}[label=(\arabic*)]
        \item 最も必要な強さが、確率24\%で軸性、29\%で硬度、18\%で耐食性、22\%で破断強度、7\%でその他であるらしいとの情報を入手しました。どちらの材料を選びますか。
        \item 全くそういった情報が入手できない場合、どちらの材料を選びますか。
    \end{enumerate}
\end{simQ}

\begin{simA}
    \begin{enumerate}[label=(\arabic*)]
        \item まずは簡単に考えてみましょう。捉性と耐食性は材料Aがよいので、足して確率 $24+18=42\%$で材料Aが有利です。他方、硬度と破断強度は材料 B がよいので、足して$29+22=51\%$で材料Bが有利です。$51\%>42\%$ですから、材料Bを選ぶほうが確率論的には正しいと言えます。
        
        \item 情報は、\cref{tab:mat}に限られます。平均がわかりやすい指標になりますが、あいにく同じですね1上。しからば、例えば「強さの最大値」と「安定した強さ」のどちらかを優先させることにしましょう。
        
        \quad 前者ならば強さ 75 を硬度でマークした材料 B を、後者ならば強さ
        間のバラつきが小さい材料 A を選ぶべきでしょう。了臨機応変に判断
        しながら、納得することが重要です。
    \end{enumerate}
    
    {\centering
    \includegraphics[width=.9\textwidth,height=.56\textwidth]{example-image-a}
    \captionof{figure}{強さ試験の結果\label{fig:result}}
    }
    
\end{simA}

\whosays{bear}{\sfcref{fig:result}に、強さ試験の結果を示します。数値をグラフにすると見た目でよくわかりますよ。}


\section{データの流れ}
上記の通り決断するためには納得感、言い換えれば理屈を必要とします。\uemph{確
率統計は、 不確かなことを決断するための理屈を提供してくれる「道具」です}。

確率統計により、情報を数学的に処理して推論を導き出します。推諭するた
めの定量的な情報を、特に\bemph{データ}と呼ぶことにしましょう。

この節では、データを取得してから何らかの結論を出すまでの過程を大まか
に把握しましょう。{\sfcref{fig:flow}}(次ページ) に、データの流れを示します。

\begin{enumerate}[label=\protect\circled{\arabic*}]
    \item データ取得 : 必要なデータとその高品質な取得方法を熟考します。
    \item データ管理 : 保管や処分のやり方やルールなどを議論し、いつでも安全に使えるようにします。
    \item データ分析 : 取得したデータの特徴を求め、そこから隠れた真実を推定
    します。  
    \item 結論導出: 仮説、条件、前提等を踏まえて、推定内容を正確に理解した
    うえで、覚悟を決めて判断をします。
\end{enumerate}

それぞれの過程で実施する詳細な内容については、後続の節で述べます。

細かい作業を実際にしていると、その意義や目的を見和拓ってしまうことも出てきます。具体的なデータ処理を行っているときにも、自分がいま行っている処理が、全体の流れのどの部分で、どんな役割を持っているのかを常に意識するようにしましょう。そうすれば、データの取り扱い能力もる向上します。

\begin{figure}[ht]
    \includegraphics[width=\textwidth,height=.95\textwidth]{example-image-a}
    \caption{データの流れ\label{fig:flow}}
\end{figure}


\section{データの取得}
取得したデータから、 まずは現状を把握します。続いて、 未来を推定 (予測、予見) します。これらの作業には、何より高品質なデータが不可欠です。たとえ、深層学習ができるニューラルネットワーク7のような優れた道具を用いたとしても、高品質なデータがなけれはばは高精度の推定は不可能です。

ビッグデータ時代の今日、\uemph{高品質なデータを持つ意義は大きい}です。 無駄なく効率よく高品質のデータを得ることに、多くの人が知恵を傾けています。

\subsection{目的データ}
時間と費用を掛けて調査をする際には、調査目的を検討、理解し、最終的にどのようなデータを取得したいのかがブレないようにすることが肝心です。最終的に取得したいデータを\bemph{目的データ}と言います。目的データとは結論を導く ために必要なデータです。 

例えば、あなたのグループで、メンバーの健康度に基づき、今後の活動を 検討することにします。この場合、健康度が目的データとなります。ここで、 データを提供するメンバーは\bemph{被験者}と呼ばれます。

なお、データを物や事象から得ることもあります。

\subsection{主観調査と客観調査}
 健康度には心理的な側面と生理的な側面があり、それらが一致しないこともあります。そこで、両方を調査する必要があります。
 
 心理的健康度は、被験者に尋ねることでわかります。具体的には、{\sfcref{fig:invs}}の ような目盛りを用意し、チェックしてもらうような調査になります。このような被験者に直接聞く調査を\bemph{主観調査}と言います。主観調査で得たデータには、真偽を確かめる有効な手段が乏しく、\uemph{信じるしかない}という特徴があります。
 
\begin{figure}[h]
     \centering
     \includegraphics[width=.4\textwidth,height=.15\textwidth]{example-image-a}
     \caption{データの流れ\label{fig:invs}}
 \end{figure}

生理的健康度は、血圧や肥満度などの測定値や、BMI 指数などの指標値で表されます。BMI 指数は、身長と体重から簡便に計算できるので、精度は落ちますが多用されています。これらの測定値や指標値は、\uemph{誰が調査してもほぼ同じになる}という特徴があり、このような調査を\bemph{客観調査}と言います。

\subsection{直接データと間接データ}
次に、データを別の視点からみてみましょう。

身長と体重などの測定データは、それ自体を直接取得できるので\bemph{直接データ}と言います。直接データは、調査しさえすれば\uemph{確実に取得できます}。

対して、BMI 指数のように別の情報から計算されるデータもあります。こ のデータは直接取得できないので、\bemph{間接データ}と言います。\uemph{間接データの精度 は、 元データの精度を上回ることはあり得ません}。したがって、統計調査に おいては、できるだけ直接データを取得すべきです。

\begin{figure}[h]
    \centering
    \includegraphics[width=.6\textwidth,height=.5\textwidth]{example-image-a}
    \caption{データの分類\label{fig:category}}
\end{figure}

\subsection{予定データと予定外データ}
また、別の分類をします。予定して得たデータを\bemph{予定データ}、そうでない データを\bemph{予定外データ}と言います。

先の例では、BMI 指数を求めるために身長と体重を調査したのだから、こ れらは全て予定データです。統計調査を成功させるには、しっかり計画し、予 定データを過不足なく取得することが肝心です。被験者への負担や分析の手間 を考えて、\uemph{取得すべきデータを厳選しましょう}。

しかし、最初から目的データが明確とは限りません。この場合には、次の予 定外データをとります。

まず、目的データに近いと思われるデータを目指して、いろいろとデータの 収集や処理を試みます。このようにして収集や処理されたデータを、かゆい部 位に手が届かずかけないみたいなので、 \ruby[g]{隔靴}{かくか}データと呼ぶことにしましょう。 かくか また、データ処理をしているうちに、有益なデータの存在に気づくこともあ ります。これを遭遇データと呼びましょう。

\uemph{本当の発見は予定外データにあることが多い}です。取得したデータを、一生懸命分析して、隠れて見えなかった予定外データを見つけ出すことが肝心です。

\begin{simQ}
    I君は、高精度と定評のある深層学習ソフトウェアを入手しました。さっそく、Web上から簡単に入手できるいくつかのデータを使って学習させました。高品質な推定結果を得られるでしょうか。
\end{simQ}

\begin{simA}
    Web上から簡単に入手したデータが高品質とは思えません。ソフトウェ アは良いに越したことはありませんが、データから導かれる推定精度を上 げるためには、データ自体の品質こそが重要です。
\end{simA}

\whosays{loli}{よい教育を受けることが大事なのは、人も機械も同じですね。}

\begin{simQ}
近年、イヤフォンが原因の難聴が、若者を中心に世界的に増加しています 。そこで、周囲のイヤフォン使用者の耳年齢について調べました。 本来、年齢相応に聞こえる周波数がありますが、耳年齢とはどの高さの周波 数まで音が聞こえるかを測定し、相応の年齢を逆算したものです。調査に 当たっては、イヤフォンをどの程度使っているかについても併せて尋ねる ことにします。

\begin{enumerate}[label=(\arabic*), itemsep=0pt, parsep=0pt]
    \item 耳年齢は直接データと間接データのいずれでしょうか。 
    \item 耳年齢は目的データでしょうか。 
    \item 「イヤフォンをどの程度使っていましたか」という尋ね方は適切でし\nobreak ょうか。
\end{enumerate}
\end{simQ}
\begin{simA}
    \begin{enumerate}[label=(\arabic*), itemsep=0pt, parsep=0pt]
        \item 間接データ。聞こえた音の周波数が直接データです。
        \item 目的データです。イヤフォンをどの程度使っているかも目的データです。
        \item 不適切です。「毎晩寝る前に少し」「う~ん、あまり」「塾の行き帰りにたまに」「仕事の必需品だ」などのさまざまな回答が出てきて、調査後の処理に困ります。
    \end{enumerate}
\end{simA}

\section{データ倫理}
\subsection{データの管理}
ータを取得したら、それを使える状態に保持しつつ、特に個人情報が漏洩 しないように管理しなければなりません。さらに、万一漏洩しても内容がわか らないように暗号化する、個人名を切り離すなどの工夫を施します。 

一方、取得したデータが将来また必要になることもあります。そのときに備えて、\uemph{データをいつ、どこで、どうやって、なぜ取得したのか、どのような データなのかをできるだけ詳細に記録しておくことも大切です}。

\subsection{被験者への配慮}
個人情報を取得する際には、被験者に説明書を渡して説明し、同意を得て同意書に署名をもらう必要があります。 

情報提供の際には、アンケートへの記入や採血などのさまざまな負担が、被験者に掛かります。狭義には被験者を実際に傷つけるデータ取得方式を、広義には被験者に許容以上の苦痛を与えるデータ取得方式を、\bemph{侵襲式}と言います。対義語は非侵襲式です。基本はもちろん、\uemph{「極力、 非侵襲式で」デー タを収集するのが望ましい}と言えます。

\subsection{データに対する心構え}
身のまわりには、膨大な情報が れかえっています。\uemph{情報化社会において は、有益あるいは有害な情報を選別し、利用や棄却する能力が問われます}。情報に埋もれず、悪意のある第三者に個人情報を盗られないように身を守らね ばなりません。

取得したデータと、そこから導き出された結論には、データを取り扱った人 の価値観や人間性が表れます。誰もがデータを扱う時代ですが、データの取り 扱い方は人によって千差万別なのです。人から安心してデータを任してもらえ る人になることが大切です。 

\begin{simQ}
18 人に対して、身長、体重、自覚健康度を無記名アンケートします。

アンケート用紙の管理上の注意事項を挙げましょう。
\end{simQ}

\begin{simA}
少人数なので、筆跡などで個人を特定されない配慮が必須です。そのため にも、適切に保管します。アンケート用紙は施錠できる閉空間に入れ、伴は 責任者が持ちます。また、電子化したデータがあれば、それをパスワードで アクセス制限し、念のため暗号化します。 

一方、適切な人がデータを正しく、スムーズに使えるような配慮も必要で す。そのために、いつ、どこでなどの調査内容に関する記録も併せて保管し ます。
\end{simA}

\begin{simQ}
次の各調査が侵襲式かどうかを考えましょう。
\begin{enumerate}[label=(\arabic*), itemsep=0pt, parsep=0pt]
    \item 脳波計を使って、睡眠の質を調べました。
    \item 友人のせきがひどいので、本人に断って血中酸素濃度を測定しま した。 
    \item ストレス度の測定をするために、指先から少しだけ採血しました。
    \item 人体にかかわる調査をするために、自分の骨格を CT スキャンしました。
    \item イヤフォンをこれまで何時間使ってきたかをアンケート調査しま した。 
    \item 病歴についてアンケート調査しました。
\end{enumerate}
\end{simQ}

\begin{simA}
どちらか迷うときは、侵襲式として考えて注意するのが無難です。
\medskip
\begin{enumerate}[label=(\arabic*)]
\item 侵襲式(心の作用を測定されてしまいます)。 
\item 非侵襲(指先にクリップ形の測定子を挟むだけで、危険はありま せん)。 
\item 侵襲式(採血は明らかに侵襲式です)。
\item 侵襲式(放射線を浴びてしまいます)。
\item 非侵襲式(被験者にかかる負担はアンケートに答える手間ぐらいで しょう)。
\item 回答が非強制であれば非侵襲式(治療に必要であれば、強制でも仕方 ありません……)。
\end{enumerate}
\end{simA}

\section{データの分析と考察}

\subsection{データの数学的処理}
いくら時間をかけて集めた高品質なデータを基にしても、その後の処理に信 頼性がなければ説得力はありません。つまり、誰もが立場によらず信頼することができる、客観性が担保された結果を得るために、\uemph{数学を使って処理をするのです}。統計学は、この数学的処理の方法論を掘り下げます。そしてそれを、集合論や確率論などの論理学に属する理論が援護射撃しています。

統計学を用いたデータ処理は、さまざまです。例えば、取得したデータ群の 特徴を明らかにしたり(第 3、6、7、8、12 章)、複数のデータ群の特徴を比較 したり(第 4 章)、それらの関係性を議論したり(第 4、5 章)、さらには複数 のデータ群から新しいデータを作り出したり(前述の BMI 指数)します。

実際には、調べたい対象全数(\bemph{母集団}と言います)からデータを取得できないことが大半でしょう。したがって、取得した一部のデータ(\bemph{標本集団}と言います)から対象全体を推定(第9\tildeto2章)する方法も必要です。

\subsection{結論の導出}
そして、データからわかることを総合して、場合によってはデータからは読み取れない情勢や仮説も加えながら、目的としている課題に対してどんなこと が言えるかを論理的に考察し、結論を導きます。すなわち、データを取り扱う までは\uemph{「見えなかった」知見}を、見つけ出すのです。

 データの取得から処理までが同じでも、人によって結論が異なることも少な くありません。それもまた、人間の成す人間のための行為ならではでしょう。データ処理をする人は、可能性の探索(想像)、今回達成できなかったこと(今後の課題)なども、結論とともに列挙し、それをメッセージとして残すことが 大切です。

\begin{exercise}
    \item\diamonds{1}
    「予」が付く語句について、次の各問に答えなさい。
    
    \begin{adjustwidth}{2.5em}{0em}
    予測・予見(=推定):推論により未来を論じる行為。\\
    予想・予期:ある基準に基づき期待して未来を論じる行為。\\
    予報:科学モデルに則り合理的に未来を論じる行為。 \\
    予言:\parbox[t][27pt]{\textwidth-8.5em}{ある物事を実現前に言明する行為。主として神秘的現象を対象と し、呪術や宗教に用いられる。}\\
    予知・予感:前もって認識する行為。\\
    予断:不完全な証拠に基づく未来への見解(判断)。\\
    予定:意志を伴う未来の行為。
    \end{adjustwidth}
    \begin{enumerate}[label=(\arabic*), leftmargin=2.25em, labelsep=.75em, parsep=0pt, itemsep=0pt, topsep=0pt]
        \item 上記の語句のうち、内容に対する客観的根拠がないのはどれですか。
        \item 上記の語句のうち、確率論的根拠に基づいてなされるのはどれですか。
        \item 以下の五つの行為を、左から理論的といえる順番に並べなさい。\\
        予断\qquad 予報\qquad 予測\qquad 予言\qquad 予想
    \end{enumerate}
    
    \item\diamonds{2}
    次のいくつかの語句どうしの関係について簡単に説明しなさい。
    \begin{enumerate}[label=(\arabic*), leftmargin=2.25em, labelsep=.75em, parsep=0pt, itemsep=0pt]
        \item 情報\qquad データ
        \item 予測\qquad 予報\qquad 予言\qquad 予定
        \item 主観調査\qquad 客観調査
        \item 直接データ\qquad 間接データ\qquad 予定データ\qquad 隔靴データ\\
        遭遇データ
    \end{enumerate}

    \newpage

    \item\diamonds{1}
    被験者からデータを得る際について、次の各問に答えなさい。
    \begin{exenum}
        \item 同意書には必ず署名してもらうべきか述べなさい。
        \item 自由回答肢形式の質問を 30 個用意しました。質問数が多いかどうか述べなさい。
        \item 謝礼は用意すべきか述べなさい。
        \item スマホやインターネットで回答を得ることに問題はあるか述べなさい。
        \item 被験者へのデータ分析結果の報告義務はあるか述べなさい。  
    \end{exenum}
    
    \item\diamonds{1} 
    統計学とは本質的にどのような学問と本書では解説しているか、簡潔にまと めなさい。
    
    \item\diamonds{1}
    データと真実の関係について、次の各問に答えなさい。
    \begin{exenum}
        \item 全数調査は常にできるか述べなさい。
        \item 身のまわりのデータや情報を列挙しなさい。
        \item 上記で挙げたデータや情報を、日頃それと認識して暮らしているか述べ なさい。
        \item 情報は常に真実を表していると言えるか述べなさい。
        \item 情報は常に正確に得られると言えるか述べなさい。
    \end{exenum}
    
    \item\diamonds{2}
    データ(情報)取得の際の注意事項について、次の各問に答えなさい。
    \begin{exenum}
        \item 被験者に負担が掛かるデータ(情報)の例を挙げ、なぜ負担が掛かるの かを簡単に説明しなさい。
        \item データ(情報)取得の際に重要なことを、簡単に述べなさい。
    \end{exenum}
    
    \item\diamonds{3} 
    類題1.4(10ページ)の続きで、「累積使用時間」を目的データとして、イヤフ\nobreak ォンをどの程度使っていたかを尋ねることにしました。
    \begin{exenum}
        \item 累積使用時間を目的データとしてよいか述べなさい。
        \item 累積使用時間だけで耳がどの程度傷んでいるかを論じられるか述べなさい。
        \item 累積使用時間は、直接データとして取得できるか述べなさい。
        \item 累積使用時間のデータを得るにはどのような尋ね方をすればよいか答えなさい。
    \end{exenum}

    \newpage
    
    \item\diamonds{1}
    統計調査の報告書作成に関して、次の文章の空欄を埋めなさい。
    
    [~\circled{1}~]データを得た後には、結果を[~\circled{2}~]やグラフ等を用いてわかりやす く整理します。[~\circled{1}~]データが得られなかった場合には、できるだけそれに [~\circled{3}~]データが得られるように統計学的な工夫を行います。
    
    データから[~\circled{4}~]を考察する際には、論理的に導かれる[~\circled{5}~]、[~\circled{6}~]に基 づいた可能性、単なる希望的な憶測などを明確に区別して、もともとの統計調 査目的から外れないように、言いたいこと、言えることをまとめます。また、「データが足りなかった」等の[~\circled{7}~]も記述しましょう。
    
    \item\diamonds{2}
    統計学でデータをどのように取り扱っていくかについて、次の各問に簡潔に 答えなさい。
    \begin{exenum}
        \item 収集したデータの管理の際に厳守すべきことを挙げなさい。
        \item 収集したデータの分析に関する基本理念を述べなさい。
    \end{exenum}
    
    \item\diamonds{3}
    ビッグデータを説明した次の文章を読み、各問に答えなさい。
    
    ビッグデータの定義はさまざまですが、「従来のデータベース管理システムでは記録、保管、解析が難しいような巨大なデータ群 17) 」と一般的に解釈され ています。ビッグデータは概して[~\circled{1}~]がさまざまで、時々刻々と[~\circled{2}~]が 増え続けます。従来はデータとして扱い切れなかったビッグデータを\uemph{\raisebox{.4ex}{~\circled{3}~}記録、 保管}し、\uemph{\raisebox{.4ex}{~\circled{4}~}素早く解析すること}で、斬新な事実や仕組みを見出すことができる と期待されています。
    \begin{exenum}
        \item 文中の~\circled{1}~と~\circled{2}~に適切な単語を入れなさい。
        \item Web 上のビッグデータに該当するデータを挙げなさい。
        \item 一般の生活環境中のビッグデータに該当するデータを挙げなさい。 \item 下線~\circled{3}~に「記録、保管」とあるが、Web 上のビッグデータはどこに記 録、保管されているか、述べなさい。 
        \item 下線~\circled{4}~に「素早く解析する」とありますが、そのために必要なことを挙 げなさい。 
        \item ビッグデータにあてはまるのは、次のうちどれか、述べなさい。\\
         構造化データ\qquad 非構造化データ\qquad 定型データ\qquad 非定型データ\\
         自己免疫性\qquad 環境対応性\qquad 時系列性\qquad リアルタイム性
        \item ビッグデータを運用するうえで注意すべきことを考え、述べなさい。
    \end{exenum}
    
    \item\diamonds{2}
    それぞれの場合に、データを新たに取得する具体的手法を述べなさい。
    \begin{exenum}
        \item 個人に直接尋ねる場合
        \item 結果的に個人から取得する場合
        \item 現象や社会を観察する場合
    \end{exenum}
    
    \item\diamonds{2}
    次の文献を、信頼性がある順番に並べかえなさい。\\
    査読論文(真偽の審査を受けた論文)\hfill 学会報告\hfill 新聞記事やTV情報\\
    特許\qquad 広告\qquad 行政広報\qquad 週刊誌\qquad Web 上の情報(発信者無記名)
\end{exercise}