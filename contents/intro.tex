\addtocounter{page}{2}
\chapter*{はじめに\markboth{はじめに}{}}

ビッグデータ時代、AI 第三世代などと言われていますが、人間が何をすべ きかという、人間の役目(評価基準?)が根底から変わりつつあります。 

その中で、データの取捨選択から分析評価までを適切に行う能力は、文部科 学省もうたっている通り、とても重要と言えます。他方、高校の数学 A には場 合の数と確率の話が、数学Ⅰ及び B には統計学の基礎部分の話が載っていま すが、高校ではいまだ幾何学や解析学のようには力を入れていない(入れられ ない)とも聞きます。

そこで、高校生(場合によっては中学生)から社会人までを広く対象とした、 「基本的な統計学的手法を即道具として使えるようにする」入門書を執筆する こととしました。データや AI の話に触れつつ、確率、集合の話から推定、検定 の話までの基本と本質的な考え方を網羅しました。また、読者の皆さんがその 手で確認しながら体得していく演習形式になっています。統計学ユーザーにこ れからなろうとしている老弱男女の皆さんが、取りあえずこれだけできれば、 身近なちょっとしたデータ処理は 9 割方できるようになると期待しています。

大学受験や企業内研修などにも、広く役立てて頂ければ幸甚です。 

\bigskip

本書は 12 章構成です。順番に学習してください。大学で、連続した 14 回程 度の講義で使用する場合などには、原則として毎回 1 章分を学習した後に、試 験を実施し、その結果を振り返って頂ければよいでしょう。

各章のページ数は極力合わせましたが、それでもなお、内容の違いにつき、 やむを得ずバラつきが発生しました。特に、第 12 章は初学者にはいささか高 度な内容なので、状況次第では省略して、代わりにページ数の多い章を 2 回分 に分割して頂ければよいと思います。

各章とも、必要に応じて「例」を挙げて原理や概要を解説し、「類題」でそれを確認しながら補強しています。そして、「Excel の問題」で実際に読者の皆様 に PC を使って数値を取り扱って頂く構成になっています。例と類題は有機的 に配列していますので、飛ばさずに学習してください。

また、学習した内容の定着に役立つよう、各章とも、章末にできるだけたく さんの「練習問題」を用意しました。 の数は難度を示します。簡単な問題か ら頑張って挑戦してみてください。

\bigskip

統計学(science of statistics)は、見えている一部のデータから、見えてい ない真実を予測するための道具だと考えます。 

予測するには、単なる希望や憶測ではいけません。そこに論理展開が必要です。集合や確率の考え方は、基盤理論としてそれを支えます。 

また、学問と言わずに道具と言ったのは、使いこなせて初めて意味があると いう思いがあります。つまり、知っているのではなく、使えることが重要なの です。道具は、良いに越したことはありませんが、たとえ、その道具がそこそ こでも、それを使いこなせたときは、おそらく良い道具を使いこなせなかった ときより良い効果を得られるものと確信しています。つまり、統計学を学ぶ目 的は、統計学を道具として使いこなせるようになることと言えます。 

ところで、見えない真実は、とても貴重、……宝物のようです。データの中に宝物が隠れていることが、たくさんあるはずです。統計学は宝探しと言えま すね。さあ、この本を読んで、隠れた宝を探し出しましょう。

\bigskip

最後に、本書の編集・組版を頂いた Green Cherry、それと出版の協力を頂 いたオーム社に謝意を表します。

\bigskip

{\small 令和 2 年 3 月}

\begin{flushright}
    \large 菱田\quad 博俊
\end{flushright}

\begin{adjustwidth}{1.5em}{0pt}
    \small\sffamily
    工学は、\\
    \mbox{\quad}人間社会の為に有意義な物や仕組みを、創出する\\
    \mbox{\qquad}学問です。 
    
    \bigskip
    
    \noindent 確率統計学は、\\
    \mbox{\quad}それを実現する為の助けとして、\\
    \mbox{\qquad}見えている現象から見えない真理に少しでも近づく為の\\\mbox{\quad\qquad}道具です。
\end{adjustwidth}