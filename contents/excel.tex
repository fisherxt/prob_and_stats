\section*{本書中の「Excel の問題」について}
\fancyhead{}

本書中にある「Excel の問題」の内容を PC で実行したことによる直接ある いは間接的な損害に対して、著作者およびオーム社は一切の責任を負いかね ます。

この「Excel の問題」で解説している実行方法は、2020 年 3 月時点のもので す。将来にわたって保証されるものではありません。 Excel は頻繁にバージョンアップがなされています。このため、本書で解説 している実行方法で実行できなくなることもありますので、あらかじめご了承 ください。 本書の発行にあたって、読者の皆様に問題なく実践していただけるよう、で きる限りの検証をしておりますが、以下の環境以外では構築・動作を確認して おりませんので、あらかじめご了承ください。

\begin{itemize}
    \item PC 本体:Windows 10 Pro 64 bit(CPU:Intel Core i5、メモリ:8 GB) 
    \item Excel 環境:Microsoft Excel 2016
\end{itemize}

また、上記環境を整えたいかなる状況においても動作が保証されるものでは ありません。ネットワークやメモリの使用状況、および同一 PC 上にある他の ソフトウェアの動作によって、本書のプログラムが動作できなくなることがあ ります。併せてご了承ください。 

本書の購入者に対する限定サービスとして、本書に掲載しているソースコー ドは、以下の手順でオーム社の Web ページからダウンロードできます。 

\begin{enumerate}[label=\protect\circled{\arabic*}]
    \item オーム社の Web ページ「https://www.ohmsha.co.jp/」を開きます。
    \item 書籍検索」で『理工系のための数学入門 確率・統計』を検索します。
    \item 本書のページの「ダウンロード」タブを開き、ダウンロードリンクをク リックします。 
    \item ダウンロードしたファイルを解凍します。
\end{enumerate} 


なお、本書に掲載しているソースコードについては、オープンソースソフト ウェアの BSD ライセンス下で再利用も再配布も自由です。